

\documentclass[english,11pt]{beamer}

\DeclareMathOperator{\Cov}{Cov}
\DeclareMathOperator{\Var}{Var}
\DeclareMathOperator{\E}{\mathbb{E}}
\DeclareMathOperator{\Proba}{\mathbb{P}}

\newcommand{\Covb}[2]{\ensuremath{\Cov\!\left[#1,#2\right]}}
\newcommand{\Eb}[1]{\ensuremath{\E\!\left[#1\right]}}
\newcommand{\Pb}[1]{\ensuremath{\Proba\!\left[#1\right]}}
\newcommand{\Varb}[1]{\ensuremath{\Var\!\left[#1\right]}}

% norm
\newcommand{\norm}[1]{\| #1 \|}

\newcommand{\indep}{\rotatebox[origin=c]{90}{$\models$}}





\usepackage{mathptmx,amsmath,amssymb,graphicx,bibentry,bbm,babel,ragged2e}

\makeatletter

\newcommand{\noun}[1]{\textsc{#1}}
\newcommand{\jitem}[1]{\item \begin{justify} #1 \end{justify} \vfill{}}
\newcommand{\sframe}[2]{\frame{\frametitle{#1} #2}}

\newenvironment{centercolumns}{\begin{columns}[c]}{\end{columns}}
%\newenvironment{jitem}{\begin{justify}\begin{itemize}}{\end{itemize}\end{justify}}

\usetheme{Warsaw}
\setbeamertemplate{footline}[text line]{}
\setbeamertemplate{headline}{}
\setbeamercolor{structure}{fg=purple!50!blue, bg=purple!50!blue}

\setbeamersize{text margin left=15pt,text margin right=15pt}

\setbeamercovered{transparent}


\@ifundefined{showcaptionsetup}{}{%
 \PassOptionsToPackage{caption=false}{subfig}}
\usepackage{subfig}

\usepackage[utf8]{inputenc}
\usepackage[T1]{fontenc}

\usepackage{multirow}

\usepackage{mdframed}


%\AtBeginSection[]
%{
%  \begin{frame}
%  \frametitle{}
%  \tableofcontents[currentsection]
%  \end{frame}
%}

\makeatother

\begin{document}



\title{Quantifying the co-evolution of economic activities locations with geo-historical data: Paris, 19th century}

\author{J.~Raimbault$^{1,2,3,4, \ast}$ and J.~Perret$^{1}$\\
$\ast$\texttt{juste.raimbault@ign.fr}
}


\institute{$^{1}$LASTIG, Univ Gustave Eiffel, IGN-ENSG\\
$^{2}$CASA, UCL\\
$^{3}$UPS CNRS 3611 ISC-PIF\\
$^{4}$UMR CNRS 8504 G{\'e}ographie-cit{\'e}s
}


\date{\textbf{Applied Urban Modelling 2022 Symposium}\\
Session 8: \\
30/06/2020
}




\frame{\maketitle}


\section{Introduction}


\sframe{SoDUCo (Social Dynamics in Urban Context) ANR project}{

\centering

\includegraphics[height=0.9\textheight]{figures/soduco_presentation.png}


}


\sframe{Structure of the SODUCO project}{

\centering

\includegraphics[height=0.9\textheight]{figures/soduco_structure.png}

}

\sframe{Project research objectives}{

% from slide kickoff
% 1. Identification et qualification des sources pertinentes Catalogue de sources primaires
%Qualification des sources et construction de méta-données
%Publication en ligne en (linked) Open Data Modélisation des imperfections associées
%2.1 et 2.2 Numérisation des sources
%Thèse avec EPITA sur l’extraction semi-automatique du contenu
%2.3 Géocodage des adresses
%2.4 Validation, correction et enrichissement collaboratifs 3. Analyse des co-évolutions
%Cartographie et géovisualisation des données et dynamiques associées
%Plateforme ouverte collaborative permettant la traçabilité des processus et des données 2 ans d’Ingénieur de recherche


}

\sframe{Verniquet atlas accuracy}{

% other work packages?
% one slide on Verniquet georef - uncertainty: cool!




\begin{columns}
	\begin{column}{0.6\textwidth}
		\centering
		\includegraphics[width=\linewidth]{figures/verniquet.png}
	\end{column}
	\begin{column}{0.4\textwidth}
		
	\end{column}
\end{columns}


}

\sframe{Vectorisation of historical maps}{

% examples yizi thesis?

}


\sframe{Quantification of intra-urban co-evolutionary dynamics}{

% this contribution research question / context 

%Urban systems are highly complex, what poses issues to ensure their resilience and sustainability and plan future cities [1]. One aspect of this complexity is their multidimen- sionality, even within subsystems such as the intra-urban location of economic activities. Indeed, different types of activities have specific location processes, yielding various patterns for their accessibility for example [2]. Understand- ing past dynamics of such social processes is crucial to build better urban models, theories and in practical terms insights for sustainable planning [3].


% [1] Michael Batty. Inventing future cities. MIT press, 2018.
%[2] Antonio Paez. Network accessibility and the spatial distri- bution of economic activity in eastern asia. Urban Studies, 41(11):2211–2230, 2004.
%[3] Celine Rozenblat, Denise Pumain, and Elkin Velasquez. In- ternational and transnational perspectives on urban systems. Springer, 2018.



}


\sframe{Urban systems and geo-historical data}{

% While recent and current urban dynamics are more and more easily tracked and quantified through the emergence of large urban data [4], historical dynamics on long time scales or on timeframes in a distant past are difficult to quantify due to the sparsity and heterogeneity of geo-historical data. At the macroscopic scale of urban systems, major transitions of past settlements systems have been modeled from an in- terdisciplinary perspective [5], and simulation models cap- turing various dimensions of systems of cities have been de- veloped [6]. At the mesoscopic and microsocpic intra-urban scales, several issues are encountered when trying to build consistent dataset, such as geocoding [7] or vectorisation to reconstruct the dynamics of road networks [8].

%[4] Jens Kandt and Michael Batty. Smart cities, big data and ur- ban policy: Towards urban analytics for the long run. Cities, 109:102992, 2021.
%[5] Lena Sanders. Peupler la terre: De la pre ́histoire a` l’e`re des me ́tropoles. Presses universitaires Franc ̧ois-Rabelais, 2018.
%[6] Denise Pumain and Romain Reuillon. Urban dynamics and simulation models. Springer, 2017.
%[7] Remi Cura, Bertrand Dumenieu, Nathalie Abadie, Benoit Costes, Julien Perret, and Maurizio Gribaudi. Historical col- laborative geocoding. ISPRS International Journal of Geo- Information, 7(7):262, 2018.
%[8] Hanae El Gouj, Christian Rincon-Acosta, and Claire Lagesse. Urban morphogenesis analysis based on geohistorical road data. Applied Network Science, 7(1):1–26, 2022.



}


\section{Data}

\sframe{Data extraction}{

% This contribution builds on data produced by the Soduco research project [9] to investigate co-evolutionary dynamics in the location of economic activities. More precisely, we focus on the case of Paris in the middle of the 19th cen- tury. Using public domain scans for main economic activ- ity repertoires (“Didot-Bottin”), an other work package of the project focused on building geo-referenced data contain- ing activities of various professionals, using Optical Char- acter Recognition techniques. We work on a sample of this data currently available, spanning 4 years between 1841 and 1844.

% -> example of Didot?

%[9] Social dynamics in urban context. Open tools, models and data - Paris and its suburbs, 1789-1950. https://soduco.github.io/.


\begin{columns}
	\begin{column}{0.6\textwidth}
		\centering
		\includegraphics[width=\linewidth]{figures/didot.png}
	\end{column}
	\begin{column}{0.4\textwidth}
		
	\end{column}
\end{columns}




}


\sframe{Data pre-processing}{

% Starting from an initial dataset of 415,976 entries, we keep the ones with correct geographical coordinates (33%), and apply natural language processing (stemming and stop-word removal) to raw strings describing activities. From there, we focus on the stems with more than 100 occurences (578 stems), and code manually a broad category of economic activity for each (obtaining the coarse grain classification within: food, craftsmanship, art and literature, health, law and governance, service, teaching). We end at this stage with 36,072 entries with identified coordinates and broad ac- tivity. Source code and results are openly available on the git repository of this work at https://github.com/ JusteRaimbault/HistoricalData.




}


% includes "Methods"
\section{Results}

\sframe{Location of activities}{

% maps - to be redone !
% stored processed data? see other ordi tonight

}


\sframe{Definition and characterisation of co-evolution}{

% To study co-evolutionary dynamics, we use the defini- tion and characterisation method proposed by [10], which is based on weak Granger causality: two urban attributes will be co-evolving if they statistically exhibit a circualar causal relationship in this context.
% -> diagram soutenance? -> slides AG soduco

%[10] Juste Raimbault. Characterising and modeling the co- evolution of transportation networks and territories. arXiv preprint arXiv:2110.15950, 2021.



}


\sframe{}{

% Aggregating spatially on raster cells (10x10 grid for the whole Paris), we thus com- pute lagged correlations between the variations of activity counts between successive dates for each cell, for each pair of activity. 

}


\sframe{}{

%We find several significant correlations (using Fisher confidence intervals), mostly negative for lagged cor- relations. This corresponds to a dynamic of sustitution of activities, with clusters rapidly replaced. Food and health are positively correlated and in co-evolution, while the other significantly co-evolving couple of activities is health and art but in a negative manner: in some districts medical pro- fessions replace artists and the contrary in others.

}

\sframe{}{

%  We also consider simultaneous correlations, and find for example that food and craftsmanship have joint dynamics in the last time interval, but not during the first considered. 

}

\section{Discussion}

\sframe{Discussion}{

% Altogether, these results first confirm the existence of a co-evolution be- tween some activities, unveil a precise characterisation of intra-urban socio-economic dynamics, and open the path to- wards more advanced thematic interpretations within an in- terdisciplinary context, such as with historians.
% Current and future work also include (i) the extension of this study on longer time spans; (ii) the combination of Granger causality with geographically weighted regression, to optimise spatial neighbourhood considered in regressions [11], and (iii) the benchmark of methods to characterise co- evolution (including e.g. instrumental variables or causal machine learning methods) on this particular dataset.

%[11] Chris Brunsdon, Stewart Fotheringham, and Martin Charlton. Geographically weighted regression. Journal of the Royal Sta- tistical Society: Series D (The Statistician), 47(3):431–443, 1998.

}




\sframe{Conclusion}{

$\rightarrow$

\medskip

$\rightarrow$


\bigskip
\bigskip
\bigskip

\textbf{Soduco repository: }\\
\url{}

\bigskip

\textbf{Models and results open at }

\url{https://github.com/JusteRaimbault/HistoricalData}

}









%%%%%%%%%%%%%%%%%%%%%
\begin{frame}[allowframebreaks]
\frametitle{References}
\bibliographystyle{apalike}
\bibliography{biblio}
\end{frame}
%%%%%%%%%%%%%%%%%%%%%%%%%%%%





\end{document}













